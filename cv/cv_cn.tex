\documentclass[margin]{res}
\usepackage{xltxtra,fontspec,xunicode}
\usepackage[slantfont,boldfont]{xeCJK}
\usepackage{zhspacing}
\newsectionwidth{17ex}

\setCJKmainfont[BoldFont={SimHei}]{SimSun} % 缺省中文字体
\setCJKmonofont{WenQuanYi Micro Hei Mono} % 设置等宽字体
\setmainfont{Liberation Sans} % 英文衬线字体
\setmonofont{Liberation Mono} % 英文等宽字体
\setsansfont{Liberation Sans} % 英文无衬线字体
\setCJKfamilyfont{Hei}{WenQuanYi Zen Hei}

\setlength{\parindent}{0in}
\setlength{\voffset}{0.1in}
\setlength{\paperwidth}{8.5in}
\setlength{\paperheight}{11in}
\setlength{\headheight}{0in}
\setlength{\headsep}{0in}
% \setlength{\textheight}{11in}
\setlength{\textheight}{12.5in}
\setlength{\topmargin}{-0.7in}
\setlength{\textwidth}{6.5in}
\setlength{\topskip}{0in}
\setlength{\oddsidemargin}{-0.7in}
\setlength{\evensidemargin}{-0.7in}



\newcommand{\ustc}{中国科学技术大学}
\newcommand{\taomee}{上海淘米网络科技有限公司}
\newcommand{\ustcsz}{中国科学技术大学苏州研究院}
\newcommand{\itemmark}{$\bullet$\mbox{~~}}


\begin{document}

{\bfseries \huge 杨思敏}
\hfill
\begin{tabular}{l}
    手机: 186\texttt{-}2180\texttt{-}9050\\
    邮箱: {smyang.ustc@gmail.com}
\end{tabular}
\rule{\columnwidth}{1pt}

\section{工作经历}
{\bf \taomee},Linux C/C++ 软件开发工程师 \hfill {\bf 2010年7月至今}\\[1mm]
\begin{itemize}
    \item 全面负责内部运维系统的后台研发工作,实现了对公司线上500台服务器和交换机的资产管理、状态监控、异常告警
    \item 
    \item 参与中国首款儿童社区游戏的后台开发工作,并逐步成长为主程
    \item 作为

\end{itemize}

\section{教育背景}
{\bf \ustc}\\[1mm]
\begin{itemize}
    \item {\bf 硕士,计算机软件与理论},计算机科学与技术学院 \hfill 2007年9月 - 2010年6月
    \item {\bf 学士,信息与计算科学},数学系 \hfill 2003年9月 - 2007年6月
    \item {\bf 双学位,金融学},统计与金融系 \hfill 2005年1月 - 2007年6月
\end{itemize}


%\section{Experience}
%\textbf{USTC-Yale Joint Research Center for High-Confidence Software, USTC.} {\em Research Assistant}\\[2mm]
%{\em \large Certifying Compiler for a C-like Programming Language} \hfill {\bf Jul 2008 - Oct 2009}\\
%\hspace*{\fill} {\em Funded by National Natural Science Foundation of China}\\
%When programming, programmers will write the corresponding program specifications to describe the behavior of the program. With our certifying compiler, the source codes and the specifications will be compiled into the executable file and the proof. Users use the proof checker to check the proof, and then indirectly check the reliability of the program. All that users have to trust is only the proof checker.
%\\[-2.5mm]
%\begin{itemize}
%\item Join in the back-end development, responsible for the conversion from AST to x86, the formal description of our abstract machine and the proof of the soundness of our system.

%\item Designed and implemented the automatic proof generator of assembly - level in cooperation with others. Responsible for writing and debugging the general-purpose proof scripts for the x86 assembly instructions and successfully reduced the size of the generated proof by 10+ times.
%\end{itemize}
%\vspace{1mm}

%{\em\large Automated Theorem Proving in Building High-confidence Software} \hfill \textbf{Jan 2009 - Aug 2009}\\
%\hspace*{\fill} {\em Funded by National Natural Science Foundation of China}\\
%At present, the verification conditions(VC) are often proved by hand, while most of automatic theorem provers only report the result but do not support the generation of the machine - checkable proof. If the automated theorem proving could be applied in program verification and implemented the prover that can automatically solves the VC and generates the machine - checkable proof, it will greatly simplify the proving of VC and broaden the application of the high confidence software.
%\\[-2.5mm]
%\begin{itemize}
%\item Proposed an innovative method to manage the information about proving, including the proof segments and the records of proving actions, and then build the machine - checkable proof terms from the records; implemented all modules of the method.

%\item Based on the improved Simplex algorithm and the above - mentioned method, designed and implemented an automated theorem prover for linear integer arithmetic, which outputs proof terms that can be checked by the proof assistant - Coq.
%\end{itemize}
%\vspace{1mm}

%{\em Teaching Assistant}
%\vspace{1mm}
%\begin{itemize}
%\item Discrete Mathematics and Its Applications, {\em School of Software Engineering}, USTC  \hfill {\bf 2008 - 2009, fall}

%\item Principles of Compiler, {\em School of Computer Science}, USTC  \hfill {\bf 2007 - 2008, spring}
%\end{itemize}
%\vspace{2mm}

% {\bf USTC E-Business Technology Co., Ltd.} {\em Part-time intern}
% \hfill {\bf Feb 2007 - May 2007}
% \vspace{1mm}
% \begin{itemize}
% \item Group member of a J2EE-based project for building a web portal platform that provides customized features and personalized content.
% \item Using BEA WebLogic Portal, achieved individuality and customization.
% \end{itemize}

\section{荣誉} 
{\bf \taomee}\\
\begin{itemize}
    \item {\bf 最佳新人奖} \hfill {\bf 2010年}
\end{itemize}
{\bf \ustc}\\
\begin{itemize}
    \item 优秀学生奖学金 \hfill {\bf 2005年}
    \item 三浦奖学金 \hfill {\bf 2004年}
    \item 优秀新生奖学金\hfill {\bf 2003年}
\end{itemize}

\section{专业技术}
\begin{itemize}
    \item CET-6. 较好的英文读写能力;
    \item Experienced with automated theorem proving and compiler;
    \item Coding experience with functional programming (SML/OCaml), Coq, C and so on;
    \item General knowledge about Linux, server and others.
\end{itemize}

%\section{Self Evaluation}
%\begin{itemize}

%\item Conscientiously, absorbedly work in a planned and order way; emphasize details;
%\item A quick learner to new ideas and new technologies;
%\item Cooperate well with colleagues, and enjoy discussing and sharing of research ideas.

%\end{itemize}

\end{document}
